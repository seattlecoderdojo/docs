% content (c) 2018 chris wallace under Creative Commons Share Alike (CC BY-SA 4.0)
% a handout to help someone get set up on Cloud 9.
\documentclass{article}

\usepackage[hidelinks]{hyperref}
\usepackage[top=1in, bottom=1in, left=1in, right=1in]{geometry}
\usepackage{multicol}

\title{Cloud 9 Setup Instructions}
\author{}
\date{}

\begin{document}
\maketitle
\thispagestyle{empty}

\section*{The Seattle Coder Dojo Workspace}
We're trying out something new this year: many of our classes will be using
something called \textbf{Cloud9}. This is a development environment, sold as a
service by Amazon Web Services (AWS). It is, however, free for students, no
credit card required.

The Cloud9 environment will give you free access to a computer that exists ``in
the cloud''. It will contain all the software you need for this class, and it
is accessible from anywhere you can get a web browser.

\section*{Getting an Account}
\begin{enumerate}
    \item Ask a mentor or volunteer about getting a Cloud9 account.
    \item Enter your email address and wait for the invitation email from \url{support@c9.io}.
    \item Click the sign up link.
    \item Create an account:
        \begin{itemize}
            \item Enter your name (or parent's name if preferred.)
            \item Enter a unique user name.
            \item Under \textit{What kind of developer are you?} choose \textit{Student}.
            \item Under \textit{How will you use Cloud9?} choose \textit{Coursework}.
        \end{itemize}

    \item Now, check your email as you should have an invitation to change your
        password. I encourage you to save this email, as it contains your
        user name.
\end{enumerate}

\section*{Creating your Workspace}
\begin{enumerate}
    \item Click \textit{Create a new Workspace}.
    \item Fill in the details:
        \begin{itemize}
            \item For \textit{workspace name} type \texttt{vim-class}
            \item For \textit{workspace description} add a meaningful description.
                \footnote{
                    In software engineering, it's expected we use
                    descriptions that explain what things are for, so others can
                    understand what we were up to.
                }
        \end{itemize}

    \item Under \textit{Templates}, select \textit{Blank}.
    \item Create your workspace!
\end{enumerate}
\end{document}
