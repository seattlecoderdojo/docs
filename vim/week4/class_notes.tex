% content (c) 2018 chris wallace under Creative Commons Share Alike (CC BY-SA 4.0)
% class notes from vim week 4
\documentclass{article}

\usepackage[hidelinks]{hyperref}
\usepackage[letterpaper, margin=1in]{geometry}
\usepackage{multicol}
\usepackage{amsmath}
\usepackage{listings}
\usepackage{paralist}
\lstset{language=bash}
\lstset{basicstyle=\footnotesize\ttfamily,breaklines=true}

\title{Vim class notes}
\author{Chris Wallace}
\date{Week 4}

\begin{document}
\maketitle

\noindent Recap: What did we talk about last week?
\begin{compactitem}
    \item conditionals
\end{compactitem}

\section*{Other Comparators}
In our previous classes, we have checked for \textit{equality} using
\texttt{-eq}. However, there are other comparison operatiors too:

\begin{center}
    \begin{tabular}{l|c|c}
        \textbf{english} & \textbf{math} & \textbf{bash} \\ \hline
        equals & $=$ & \texttt{-eq} \\
        not equals & $\neq$ & \texttt{-ne} \\
        less than & $<$ & \texttt{-lt} \\
        greater than & $>$ & \texttt{-gt} \\
        less than or equal to& $\le$ & \texttt{-le} \\
        greater than or equal to& $\ge$ & \texttt{-ge}
    \end{tabular}
\end{center}

\section*{Loops}
Just like conditionals, all languages have loops. It's something that allows us
to run the same code multiple times.
\begin{lstlisting}
i=1
while [ $i -lt 5 ]
do
    echo "$i"
    (( i++ ))
done
\end{lstlisting}

You can also use loops to do something over and over until a human stops it.
\begin{lstlisting}
times=1
go=1
while [ $go -eq 1 ]
do
    echo "we went $times times"
    (( times++ ))
    read -p "should we continue? press 1: " go
done
\end{lstlisting}

\section*{Data Types}
What are data types? Just the types of data we have! We discussed that, in this
class, we've primarily dealth with three: Booleans, Integers, and Strings.

\subsection*{Booleans}
Booleans have two possible values: \textbf{True} or \textbf{False}. Each one of
our \textbf{if} statements dealt with booleans.

\subsection*{Integers}
If you remember from math class, integers are positive or negative whole
numbers, including zero. Some examples are -1, 0, 1, 2, etc\dots

For the shell/bash programming language, the comparisons we've learned
(\texttt{-eq}, \texttt{-lt}, etc are all operations on integers. You couldn't
use these to compare strings, which we'll discuss next.

\subsection*{Strings}
Strings are the types of data we keep in the \texttt{"} (quotation) marks. \texttt{"hi"} is a string in bash.

\subsection*{Examples}
Some programming languages make you specify the data type of a variable you
even name it! However, bash is not one of those languages- it's up to you to
look at the variable and think about the type of data it is.

\begin{lstlisting}
# integers
x=4
y=7
z=-1

# strings
name="chris"
greeting="Hello there, human"
stringnum="5"

# convert x ond y to strings
x="$x"
y="$y"
\end{lstlisting}

\section*{File Permissions}
Finally, we discussed that you can run your program directly by doing:
\begin{lstlisting}
./program_name.sh
\end{lstlisting}

However, we noted that there are two prerequisites:
\begin{enumerate}
    \item Your first line of the file is \lstinline|#!/bin/bash|
        \begin{itemize}
            \item This says that "Hey, this file is meant to be run by this program"
            \item \lstinline|/bin/bash| is where \lstinline|bash| lives.
        \end{itemize}

    \item You have \textit{execute permissions} to this script.
        \begin{itemize}
            \item We used the command \lstinline|chmod u+x ./program_name.sh|

            \item This means ``give \textbf{U}ser the e\textbf{X}ecute
                permissions to this file.''
        \end{itemize}
\end{enumerate}

Here's example output of the command \lstinline|ls -l|
\begin{verbatim}
total 12
--w------- 1 ubuntu ubuntu  48 Jan 21 00:48 no_read.txt
---x--x--x 1 ubuntu ubuntu   9 Jan 21 00:51 no_read_no_write.txt*
-r--r--r-- 1 ubuntu ubuntu 177 Jan 28 23:25 read_no_write.txt
\end{verbatim}

Looking at the left, we explained that there are 10 ``spots''. Each one of
these spots is either ``occupied'' with a letter, or is a \texttt{-}. To phrase
it differently, each file has 10 \textit{boolean} spots about their permissions.
\begin{itemize}
    \item The first spot is \texttt{d}, which will be set if this is a directory.
    \item The next 3 spots are for the \textbf{user's} read, write, and execute permissions.
    \item The next 3 spots are for the \textbf{group's} read, write, and execute permissions.
    \item The next 3 spots are for everyone else, or \textbf{other's} read, write, and execute permissions.
\end{itemize}

This command also lists the owner (ubuntu), the owning group (ubuntu), and the
last modified time of the file in UTC time.

We can use the \textbf{chmod} command to add or remove permissions.

\end{document}
