% content (c) 2018 chris wallace under Creative Commons Share Alike (CC BY-SA 4.0)
% the project: write rock/paper/scissors in vim
\documentclass{article}

\usepackage[hidelinks]{hyperref}
\usepackage[top=1in, bottom=1in, left=1in, right=1in]{geometry}
\usepackage{multicol}
\usepackage{listings}
\lstset{language=bash}
\lstset{basicstyle=\footnotesize\ttfamily,breaklines=true}

\title{Rock Paper Scissors Script}
\author{}
\date{Revision 2}

\begin{document}
\maketitle
\thispagestyle{empty}

Rock paper scissors is a game played between two people, where each person
makes a choice between rock, paper, or scissors. This game has clear rules-
rock beats scissors, paper beats rock, and scissors beats paper.

As this game has clear winning conditions and simple inputs (only two players,
each player can only pick one of three choices), this game can be programmed
using simple control flow.

In this project, you'll use vim to write a bash script named \texttt{rpc.sh}. This script will:
\begin{enumerate}
    \item Tells the player the rules of this game.
    \item Asks the user to choose Rock (0), Paper (1), or Scissors (2).
    \item Picks a random number for the computer's choice (0, 1, or 2).
    \item Computes and displays the winner of this competition.
\end{enumerate}

\section*{Helpful Shell Commands}
Reading into a variable, printing that variable:
\begin{lstlisting}
read var # this waits for input
echo "$var"
\end{lstlisting}
Conditionals (numbers):
\begin{lstlisting}
choice=0
if [ $choice -eq 0 ]
then
    echo "choice was rock!"
elif [ $choice -eq 1 ]
then
    echo "choice was paper!"
else
    echo "choice was not rock or paper, so it is probably scissors!"
fi
\end{lstlisting}
Get a random number between 0 and 2, and store into variable called \texttt{var}:
\begin{lstlisting}
var= eval $RANDOM % 3
\end{lstlisting}

\section*{Helpful Vim Commands}
\begin{multicols}{2}
    \begin{description}
        \item[h, j, k, l] move around (left down up right)
        \item[i] go into insert mode, \textit{at} the cursor
        \item[a] go into insert mode, \textit{after} the cursor
        \item[b, w] back a word, forward a word
        \item[:w] write (save) this file
        \item[:wq] write and quit vim
        \item[:q!] quit without saving
        \item[dw] delete forward a word
        \item[dd] delete line
        \item[2dd] delete two lines
    \end{description}
\end{multicols}

\end{document}
