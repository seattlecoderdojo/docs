% content (c) 2018 chris wallace under Creative Commons Share Alike (CC BY-SA 4.0)
% class notes from vim week 2
\documentclass{article}

\usepackage[hidelinks]{hyperref}
\usepackage[letterpaper, margin=1in]{geometry}
\usepackage{multicol}
\usepackage{amsmath}
\usepackage{listings}
\lstset{language=bash}
\lstset{basicstyle=\footnotesize\ttfamily,breaklines=true}

\title{Vim class notes}
\author{Chris Wallace}
\date{Week 2}

\begin{document}
\maketitle

First, we talked about what went well last week and what didn't. Namely, we
discussed how there are students of all levels here so we're going to go slow,
and fast, by having a project to work on soon. Additionally, we noted that the
easiest way to log into cloud9 was to visit \url{https://c9.io/login} directly.

\section*{File and Folder Manipulation}
To use commands, we first type the command name, followed by any
\textit{arguments} that we want the command to work on:
	\[
        \hspace {0em} \underbrace{\texttt{mv}}_{\text{command}}
        \underbrace{
        \hspace {1em}
            \texttt{twocities.txt dickens.txt}
        \hspace {1em}
        }_{\text{arguments}}
	\]

~\\
\noindent
We introduced the following commands for working with \textit{files}:
\begin{itemize}
    \item[ls] list a directory's contents.
    \item[cat] put files together and print them. We can use this to look at the
        contents of a file.
    \item[mv] move or rename a file.
    \item[cp] copy a file.
    \item[rm] remove a file.
    \item[touch] create an empty file.
\end{itemize}

\noindent
We also introduced the following commands for working with \textit{folders}:
\begin{itemize}
    \item[cd] or \textit{change directory}. We use this to move around folders.
    \item[mkdir] makes a directory.
    \item[rmdir] removes a directory, although only when it's empty.
\end{itemize}

When moving around folders with \texttt{cd}, we introduced the concept of
\textit{absolute} and \textit{relative} paths. Absolute paths are places
described as where they are from the root directory, or \texttt{/}. An example
would be \texttt{/home/cwwllac/Desktop/class}. We also discussed how a user's
home directory is pretty important, so we get a shortcut to it: the squiggly
tilde: (\textasciitilde{}). An example path using the tilde (which is still an
absolute path) is \texttt{\textasciitilde{}/Desktop/class}

Relative paths are simpler, they are just getting you somewhere
\textit{relative} to where you are now. If you were home, to get to
\texttt{Desktop/class} you could simply do: \texttt{cd Desktop/class}.
The two shortcuts to note when using relative paths are (\texttt{.}) and
(\texttt{..}) - a single period is the \textit{current} folder we're in, and a
double period is the folder \textit{above} (or ``back behind'') this one.

Finally, we talked about how to escape a ``stuck program'' with Control+C.

\section*{Input Redirection}
First, we downloaded a file using the following command:
\begin{lstlisting}
wget scdojo.s3.amazonaws.com/vim/twocities.txt
\end{lstlisting}
This file happens to be the book \textit{A Tale of Two Cities} by Charles
Dickens. We played around with this book with some commands. Note that any line
beginning with a \texttt{\#} is a comment and is not code.
\begin{lstlisting}
# count the number of lines, words, and characters in twocities.txt
wc twocities.txt
# search twocities.txt for any line containing the word "times"
grep "times" twocities.txt

# redirect the output of a command to a file (times.txt)
grep "times" twocities.txt > times.txt
# count the number of lines, words, and characters in that new file
wc times.txt

# redirect the output of a command to another command
grep "times" twocities.txt | wc
\end{lstlisting}
We then installed a new program called \textit{cowsay}:
\begin{lstlisting}
# update package lists. We use sudo because this requires being "super user"- the administrator
sudo apt-get update
# install cowsay
sudo apt-get install cowsay

# say hi
echo "hello world"
# say hi using a cow
echo "hello world" | /usr/games/cowsay
# save all that to a file
echo "hello world" | /usr/games/cowsay > hello_cow.txt
\end{lstlisting}

\section*{Put it in a File}
Finally, using vim, we put these things together in an actual file:
\begin{enumerate}
    \item Open vim with \texttt{vim hello\_world.sh} - we ended it in .sh because this is a \textit{shell script}
    \item Press \textbf{i} to go into insert mode
    \item Add a comment to the file (line starting with \texttt{\#}) about what this file will do
    \item Add any script you want using \texttt{echo}, \texttt{cowsay}, etc
\end{enumerate}

We also introduced variables, which are things that can hold value and even
have their value change. See a full example below:
\begin{lstlisting}
# script to say hi and ask your name
echo "Hello world!"
name="Chris"
echo "My name is $name", what is yours? (please type)"
read name
echo "Cool! Hello there $name"
\end{lstlisting}

\end{document}
